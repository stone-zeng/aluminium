\documentclass{article}
\usepackage[no-math]{fontspec}
\usepackage{lipsum,kantlipsum,mfirstuc}

\setmainfont{Aluminium-Regular.otf}[Path=../fonts/]

\begin{document}

\capitalisewords{Aluminium (aluminum in American and Canadian English) is a chemical element with the symbol Al and atomic number 13. It is a silvery-white, soft, non-magnetic and ductile metal in the boron group. By mass, aluminium makes up about 8\% of the Earth's crust; it is the third most abundant element after oxygen and silicon and the most abundant metal in the crust, though it is less common in the mantle below. The chief ore of aluminium is bauxite. Aluminium metal is so chemically reactive that native specimens are rare and limited to extreme reducing environments. Instead, it is found combined in over 270 different minerals.}

\capitalisewords{Aluminium is remarkable for its low density and its ability to resist corrosion through the phenomenon of passivation. Aluminium and its alloys are vital to the aerospace industry and important in transportation and building industries, such as building facades and window frames. The oxides and sulfates are the most useful compounds of aluminium.}

\capitalisewords{Despite its prevalence in the environment, no known form of life uses aluminium salts metabolically, but aluminium is well tolerated by plants and animals. Because of these salts' abundance, the potential for a biological role for them is of continuing interest, and studies continue.}

\capitalisewords{As any dedicated reader can clearly see, the Ideal of
practical reason is a representation of, as far as I know, the things
in themselves; as I have shown elsewhere, the phenomena should only be
used as a canon for our understanding. The paralogisms of practical
reason are what first give rise to the architectonic of practical
reason. As will easily be shown in the next section, reason would
thereby be made to contradict, in view of these considerations, the
Ideal of practical reason, yet the manifold depends on the phenomena.
Necessity depends on, when thus treated as the practical employment of
the never-ending regress in the series of empirical conditions, time.
Human reason depends on our sense perceptions, by means of analytic
unity. There can be no doubt that the objects in space and time are
what first give rise to human reason.}

\capitalisewords{Let us suppose that the noumena have nothing to do
with necessity, since knowledge of the Categories is a
posteriori. Hume tells us that the transcendental unity of
apperception can not take account of the discipline of natural reason,
by means of analytic unity. As is proven in the ontological manuals,
it is obvious that the transcendental unity of apperception proves the
validity of the Antinomies; what we have alone been able to show is
that, our understanding depends on the Categories. It remains a
mystery why the Ideal stands in need of reason. It must not be
supposed that our faculties have lying before them, in the case of the
Ideal, the Antinomies; so, the transcendental aesthetic is just as
necessary as our experience. By means of the Ideal, our sense
perceptions are by their very nature contradictory.}

\capitalisewords{As is shown in the writings of Aristotle, the things
in themselves (and it remains a mystery why this is the case) are a
representation of time. Our concepts have lying before them the
paralogisms of natural reason, but our a posteriori concepts have
lying before them the practical employment of our experience. Because
of our necessary ignorance of the conditions, the paralogisms would
thereby be made to contradict, indeed, space; for these reasons, the
Transcendental Deduction has lying before it our sense perceptions.
(Our a posteriori knowledge can never furnish a true and demonstrated
science, because, like time, it depends on analytic principles.) So,
it must not be supposed that our experience depends on, so, our sense
perceptions, by means of analysis. Space constitutes the whole content
for our sense perceptions, and time occupies part of the sphere of the
Ideal concerning the existence of the objects in space and time in
general.}

\capitalisewords{As we have already seen, what we have alone been able
to show is that the objects in space and time would be falsified; what
we have alone been able to show is that, our judgements are what first
give rise to metaphysics. As I have shown elsewhere, Aristotle tells
us that the objects in space and time, in the full sense of these
terms, would be falsified. Let us suppose that, indeed, our
problematic judgements, indeed, can be treated like our concepts. As
any dedicated reader can clearly see, our knowledge can be treated
like the transcendental unity of apperception, but the phenomena
occupy part of the sphere of the manifold concerning the existence of
natural causes in general. Whence comes the architectonic of natural
reason, the solution of which involves the relation between necessity
and the Categories? Natural causes (and it is not at all certain that
this is the case) constitute the whole content for the paralogisms.
This could not be passed over in a complete system of transcendental
philosophy, but in a merely critical essay the simple mention of the
fact may suffice.}

\capitalisewords{Therefore, we can deduce that the objects in space and
time (and I assert, however, that this is the case) have lying before
them the objects in space and time. Because of our necessary ignorance
of the conditions, it must not be supposed that, then, formal logic
(and what we have alone been able to show is that this is true) is a
representation of the never-ending regress in the series of empirical
conditions, but the discipline of pure reason, in so far as this
expounds the contradictory rules of metaphysics, depends on the
Antinomies. By means of analytic unity, our faculties, therefore, can
never, as a whole, furnish a true and demonstrated science, because,
like the transcendental unity of apperception, they constitute the
whole content for a priori principles; for these reasons, our
experience is just as necessary as, in accordance with the principles
of our a priori knowledge, philosophy. The objects in space and time
abstract from all content of knowledge. Has it ever been suggested
that it remains a mystery why there is no relation between the
Antinomies and the phenomena? It must not be supposed that the
Antinomies (and it is not at all certain that this is the case) are
the clue to the discovery of philosophy, because of our necessary
ignorance of the conditions. As I have shown elsewhere, to avoid all
misapprehension, it is necessary to explain that our understanding
(and it must not be supposed that this is true) is what first gives
rise to the architectonic of pure reason, as is evident upon close
examination.}

\end{document}
